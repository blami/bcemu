%%%%%%%%%%%%%%%%%%%%%%%%%%%%%%%%%%%%%%%%%%%%%%%%%%%%%%%%%%%%%%%%%%%%%%%%%%%%%%%
% bc.tex - Bachelor thesis                                                    %
% Subject: bc_emu - portable video game emulator                              %
% Chapter: Uvod                                                               %
% Author: Ondrej Balaz <ondra@blami.net>                                      %
%%%%%%%%%%%%%%%%%%%%%%%%%%%%%%%%%%%%%%%%%%%%%%%%%%%%%%%%%%%%%%%%%%%%%%%%%%%%%%%

\chapter{Úvod}\label{chap:intro}

Neuvěřitelné tempo, kterým se v dnešní době vyvíjejí nové technologie v oblasti
výpočetní techniky, nechává mnohem rychleji, než v jiných oblastech, zestárnout
ty včerejší. Čekání s vývojem software na finální verzi hardware dnes téměř
implikuje vývoj pro zastaralý hardware a v některých odvětvích, jako je třeba
videoherní průmysl, může mít pro autora kritické následky. Emulace je postup,
který pomáhá vývojářům software udržet krok s vývojáři hardware a umožňuje jim
snadno vytvářet software pro hardware, který ještě ve finální podobě
neexistuje. Kromě toho, že je díky emulaci paralelní vývoj software a hardware
běžnou praxí, už i někteří výrobci zpřístupňují ve svých nejmodernějších
výrobcích emulaci těch starších, které si jejich zákazníci
oblíbili.
%\footnote{Největší současní výrobci videoherních konzolí, firmy Sony a
%Nintendo, poskytují službu, v rámci které je možné pomocí emulátorů
%nainstalovaných v konzolích Sony Playstation 3 a Portable, nebo Nintendo Wii
%přehrávat i několik desítek let staré herní tituly.}.

Emulace ale není jen nástroj, který nám pomáhá dostat se s vývojem rychle
kupředu. Můžeme se díky ní ohlédnout i do minulosti, což je jistě zajímavé
právě v oblasti zmíněného herního průmyslu. Za posledních 40 let vznikl
nespočet herních titulů pro 7 generací videoherních konzolí, které si díky
nízkým pořizovacím nákladům a jednoduchosti používání našly své neohrozitelné
místo na trhu. Většina z těchto titulů dávno upadla do zapomění nahrazena
novějšími a emulace videoherních konzolí je tak, vzhledem k nedostupnosti
potřebného hardware, jedinný způsob jak si tyto tituly připomenout.

Jednou z herních konzolí, pro kterou některé z těchto titulů vznikly je NEC
PCEngine. Jedná se o poměrně revoluční koznoli 3. generace, která použitím
16-ti bitové technologie pro obrazový výstup a velice kvalitním zvukovým
výstupem předčila technické možnosti své konkurence. Snad vlivem špatného
marketingu společnosti NEC se nikdy nestala tak populární jako ostatní konzole
této generace, což je společně s tím, že společnost NEC šla cestou použití
komponent navržených výhradně pro tento systém (např. procesoru)
důvodem, proč existuje pouze omezené množství emulátorů této konzole. Právě
zajímavá architektura tohoto systému a poměrně malý počet dostupných emulátorů
jsou hlavním důvodem, proč se tento systém stal předmětem této práce.

Cílem této práce je prostudování architektury videoherního systému NEC
PCEngine, její shrnutí do ucelené technické specifikace (pravděpodobně jedinné
v českém jazyce) a následná analýza, návrh a implementace emulátoru tohoto
systému na jejím základě. Výsledný emulátor by měl, na rozdíl od většiny
existujících implementací, umožňovat snadné rozšíření o podporu dalších
videoherních systémů a být jednoduše přenositelný i na jiné platformy, než
jsou operační systémy osobních počítačů.

\begin{comment}

Vznik a existence videoherního průmyslu jsou přirozenou reakcí výrobců
elektronických zařízení na lidskou potřebu bavit se. Historie videoher sahá až
do roku 1947, kdy bylo patentováno první \uv{zařízení} tohoto druhu - analogový
obvod implementující jednoduchou hru promítanou na stíntko osciloskopu. K
prvním komerčním úspěchům v této oblasti však vedla ještě dlouhá cesta, než v
roce 1972 společnost Atari začala vyrábět legendární videohru {\em PONG}, která
svým jednoduchým principem, provedením umožňujícím provoz na běžném televizoru
a nízkou cenou zpřístupnila tento nový druh zábavy široké veřejnosti.

Popularita a úspěch videohry {\em PONG} byly důležitým impulzem pro vznik
konceptu videoherní konzole, elektronického zařízení pro domácí zábavu. Díky
jednoduchosti obsluhy, nízkým pořizovacím nákladům a rozmanité škále žánrů
herních programů si v průběhu uplynulých 40-ti let vybudovalo 7
generací\footnote{Dělení videoherních konzolí na generace se odvíjí od velkých
technických kroků v této oblasti, jako je přechod od 8-mi bitů k 16-ti atd.,
více viz.~\cite{wwwWikiConsole}} videoherních konzolí neotřesitelné místo na
trhu.

V průběhu těchto 40-ti let také vzniklo nesčetné množství více či méně
populárních herních programů, které s příchodem nových technologií, a tedy i
videoherních konzolí nových generací, zastárly. Původní hardware nutný ke
spuštění těchto programů je v dnešní době prakticky nesehnatelný a jedinnou
možností jak se k těmto programům vrátit je emulace videoherní konzole, pro
kterou byly určeny.

Cílem této práce je nastudovat architekturu videoherní konzole NEC PCEngine a
na základě 

Vrcholu své popularity dosáhly videoherní konzole v letech 1983-1995, tedy v
době, kdy už byla běžně dostupná osobní výpočetní technika, ale v případě, že
by měla sloužit pouze jako zdroj zábavy, byla její cena stále příliš vysoká.
Dostupné technologie společně s důmyslným návrhem však umožnily přinést na
trh levné, ale přesto výkonné videoherní konzole 3. generace, které se
vyznačovaly detailní a propracovanou grafikou ve vysokém rozlišení, kvalitním
zvukovým výstupem a novými způsoby ovládání.

Příchod videoherních konzolí 3. generace měl za následek nové možnosti při
vývoji herních programů, díky čemuž právě v této době vznikla řada
nezapomenutelných titulů (např. {\em Final Fantasy}, {\em Super Mario Bros.},
{\em The Legend of Zelda}, {\em Castlevania}, {\em Contra}), které možná byly
nebo budou překonány po stránce technického zpracování, ale nikoliv po stránce
originality a nápadu, o čemž svědčí i fakt, že velké množství dnešních
videoherních titulů z nich čerpá, nebo je jejich kopií oblečenou do \uv{nového
kabátu}.

Právě do 3. generace videoherních konzolí patří systém NEC PCEngine. Jeho
unikátní architektura založená na 8-mi bitovém hlavním procesoru a 16-ti
bitovém grafickém subsystému


\end{comment}
