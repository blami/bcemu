%%%%%%%%%%%%%%%%%%%%%%%%%%%%%%%%%%%%%%%%%%%%%%%%%%%%%%%%%%%%%%%%%%%%%%%%%%%%%%%
% bc.tex - Bachelor thesis                                                    %
% Subject: bc_emu - portable video game emulator                              %
% Chapter: Existujici implementace                                            %
% Author: Ondrej Balaz <ondra@blami.net>                                      %
%%%%%%%%%%%%%%%%%%%%%%%%%%%%%%%%%%%%%%%%%%%%%%%%%%%%%%%%%%%%%%%%%%%%%%%%%%%%%%%

\chapter{Existující implementace}\label{chap:exist}

I přes to, že systém NEC PCEngine nikdy nedosáhl podobné popularity a rozšíření
ve světe, jako jiné videoherní konzole stejné generace (Nintendo Famicom, Sega
Master System, Atari 7800), vzniklo několik jeho emulátorů. Jejich existence
značně usnadňuje vývoj a lazení nového emulátoru. Kromě toho, že mohou být
použity jako prostředek pro ověření funkčnosti v případě absence skutečného,
dnes již nedostupného, hardware systému NEC PCEngine, je řada z nich šířena i
ve zdrojové podobě jako svobodný software, což umožňuje studovat jak
specifikaci systému, tak i způsob jakým jejich autoři vyřešili vlastní
implementaci jeho jednotlivých částí.

V úvodní sekci~\ref{chap:exist_listing} této kapitoly je nastíněna situace v
oblasti emulace systému NEC PCEngine a jsou představeny některé existující
implementace emulátorů. Sekce~\ref{chap:exist_summary} shrnuje přednosti a
nedostatky těchto implementací a na jejich základě nastiňuje směr pro další
analýzu, návrh a implementaci vlastního emulátoru.

% -----------------------------------------------------------------------------
% Přehled emulátorů NEC PCEngine
% -----------------------------------------------------------------------------

\section{Přehled emulátorů NEC PCEngine}\label{chap:exist_listing}

Většina emulátorů systému NEC PCEngine vznikla v letech 1995-2001, hlavně pro
osobní počítače s operačními systémy Microsoft DOS a Microsoft Windows. V
následujícím přehledu jsou uvedeny pouze tři nejpoužívanější, aktivně udržované
a vyvíjené emulátory. Další, které většinou nejsou zdaleka tak kompletní z
hlediska funkčnosti a kompatibility se zveřejněnými čipovými kartami HuCard,
nebo je jejich kód zastaralý a jsou v dnešních podmínkách nepoužitelné, lze
nalézt např. na~\cite{wwwEmulatorZone}.

% MagicEngine

\subsection{MagicEngine}

MagicEngine~\cite{wwwMagicEngine} je komerční emulátor systému NEC PCEngine.
Podporuje emulaci všech verzí systému včetně řady rozšíření (např. CD-ROM
mechanika). Emulace je v podání tohoto emulátoru velice přesná, rychlá a bez
problému je možné spustit většinu obrazů zveřejněných na čipových kartách
HuCard, včetně těch, které využívají nestandardního chování systému.

Podporovány jsou platformy Microsoft DOS (jen starší verze emulátoru),
Microsoft Windows a Apple MacOS X. Program disponuje vlastním jednoduchým
grafickým uživatelským rozhraním. Nároky na hardware jsou poměrně vysoké a při
použití pod operačním systémem Windows vyžaduje MagicEngine grafickou
akceleraci pomocí DirectX nebo OpenGL. V současné době se jedná o
nejpopulárnější emulátor NEC PCEngine pro Microsoft Windows.

Přesto, že autoři MagicEngine významným dílem přispěli ke zdokumentování
architektury NEC PCEngine a zveřejnili řadu nástrojů pro vývoj programů s tímto
systémem kompatibilních, zdrojový kód MagicEngine není otevřený, což znemožňuje
zásahy kýmkoliv jiným než autory.

\subsubsection*{Klíčové vlastnosti}

\begin{itemize}
\item vysoká přensnost emulace
\item podpora všech verzí a rozšíření systému
\item kompatibilita s většinou čipových karet HuCard
\end{itemize}

% Ootake

\subsection{Ootake}

Ootake~\cite{wwwOotake} je emulátor systému NEC PCEngine od japonského autora
Nakamura Kitao. Podobně jako MagicEngine podporuje Ootake všechny verze systému
včetně řady rozšíření. Autor se speciálně věnuje problematice dosažení stejného
pocitu ze hry\footnote{Např. poměrně složitě implementuje zpracování vstupu
z herního ovladače tak, aby bylo kompenzováno zpoždění způsobené obsluhami
operačního systému.} jako na originálním hardware, takže je emulátor velice
přesný i po stránce zpracování vstupu apod. Množina podporovaných titulů je o
něco menší než v případě MagicEngine.

Ootake je napsán výhradně pro platformu Microsoft Windows v jejímž duchu se
také nese uživatelské rozhraní programu. Hardwarové nároky Ootake jsou poměrně
vysoké, emulátor vyžaduje grafickou akceleraci pomocí DirectX a procesor s
rychlostí cca 1.6GHz pro plynulý běh herních programů.

Zdrojový kód emulátoru Ootake je veřejně přístupný a šířen pod licencí GNU/GPL.
Vzhledem k tomu, že je program vyvíjen bez ohledu na přenositelnost kódu, na
řadě míst se prolíná kód zajišťující logiku emulace a specifický kód pro
uživatelské rozhraní v Microsoft Windows. Kód je z důvodu přesnosti emulace
protkán řadou optimalizací a jeho čitelnost navíc ztěžuje fakt, že veškeré
komentáře jsou psány v japonštině.

\subsubsection*{Klíčové vlastnosti}

\begin{itemize}
\item vysoká přesnost emulace s ohledem na pocit ze hry
\item podpora všech verzí a rozšíření systému
\item otevřený zdrojový kód
\end{itemize}

% Mednafen

\subsection{Mednafen}

Mednafen~\cite{wwwMednafen} je emulátor několika populárních videoherních
systémů mezi nimiž je i systém NEC PCEngine. Emulace NEC PCEngine v rámci
tohoto programu využívá několika společných komponent s emulátorem systému NEC
PC FX32, který je nástupcem NEC PCEngine. Stejně jako v předchozích případech
je podporována řada rozšíření. I přes některé nepřesnosti časování v emulaci je
celková přesnost emulace vysoká. Součástí programu je podpora hry více hráčů na
jedné konzoli přes TCP/IP (vzdálený herní ovladač), nebo vestavěný ladicí
nástroj, který umožňuje krokovat běžící herní program a sledovat obsah
významných oblastí paměti a registrů.

Program je psán přenositelně a je prokazatelně možné ho sestavit a používat na
platformách GNU/Linux, FreeBSD, Apple MacOS X a Microsoft Windows. Uživatelské
rozhraní programu je zajištěno pomocí přenositelné knihovny pro uživatelská
rozhraní libSDL. Hardwarové nároky na plynulou emulaci jsou oproti předchozím
řešením nižší.

Mednafem má otevřený zdrojový kód šířený pod licencí GNU/GPL. Je psán v jazyce
C++ s důrazem na přenositelnost, hlavně mezi platformami GNU/Linux a Microsoft
Windows. Nepočítá se však s přenositelností na platformy, kde není dostupná
knihovna libSDL, která je pevnou součástí programového kódu. Stejně tak
jednotlivé emulátory podporovaných systémů nepoužívají žádné společné rozhraní,
které by umožňovalo program jednoduše rozšířit.

\subsubsection*{Klíčové vlastnosti}

\begin{itemize}
\item podpora dalších videoherních systémů
\item integrovaný nástroj pro lazení emulovaných programů
\item podpora hry více hráčů po síti TCP/IP
\item otevřený zdrojový kód
\item přenositelnost
\end{itemize}

% -----------------------------------------------------------------------------
% Shrnuti
% -----------------------------------------------------------------------------

\section{Shrnutí}\label{chap:exist_summary}

Většina existujících implementací emulátoru NEC PCEngine (včetně řady těch,
které nejsou v předchozím výčtu uvedeny) jsou jednoúčelové programy soustředící
se výhradně na emulaci tohoto systému, případně jeho dalších variant a
rozšíření.

Jistě pozitivní stránkou tohoto přístupu je v některých případech vysoká
přesnost a kvalita emulace, protože se autor nemusí soustředit na řadu dalších
podporovaných videoherních systémů. Na druhou stranu je množina opravdu
nezapomenutelných a populárních herních titulů pro jednotlivé videoherní
systémy poměrně malá\footnote{Velice populární japonská série herních titulů
Final Fantasy vydávaná společnosti Square Enix má v současné době více než 13
dílů, které vyšly střídavě pro 12 různých typů videoherních systémů s průměrnou
bilancí 1-2 díly na jeden typ videoherního systému.} a uživatelé často
používají více emulátorů několika různých systému. Jsou tak nuceni používat
více jednoúčelových programů s různým způsobem ovládání a chováním i přesto, že
jejich úkol je z uživatelského pohledu stejný.

\uv{Slabinou} existujících implementací je také poměrně malá množina
podporovaných platforem. Jak již bylo uvedeno, většina emulátorů systému NEC
PCEngine vznikla v letech 1995-2001, tedy v období kdy poměrně vysokým nárokům
na výkon při provádění emulace dostačovaly prakticky jen osobní počítače a v
oblasti operačních systémů nebyl např. GNU/Linux zdaleka tak populární jako
Microsoft DOS nebo Microsoft Windows.

V současné době existuje řada videoherních handheldů\footnote{Přenosné kapesní
videoherní konzole vybavené displejem a ovládacími prvky herního ovladače,
často napájené z baterií, např. Sony Playstation Portable, Nintendo DS nebo
Gamepark Wiz.}, chytrých mobilních telefonů\footnote{Mobilní telefony vybavené
operačním systémem umožňujícím instalaci a spouštění dalších programů, např.
Nokia Series60 s operačním systémem Symbian nebo telefony HTC s operačními
systémy Google Android nebo Microsoft Windows Mobile} a kapesních počítačů,
které mohou výkonem směle konkurovat tehdejším osobním počítačům, nikoliv však
těm dnešním, což je jeden z důvodů proč právě tato zařízení jsou zajímavou,
platformou pro emulátory starších videoherních systémů. Většina existujících
emulátorů ale stěží počítá s přenositelností na jiné operační systémy než je
Microsoft Windows, natož na tato zařízení.

Kromě slabin existujících implmentací ale existují i jejich přednosti. Nejvíce
inspirativní je v této oblasti poslední zmíněný emulátor Mednafen, který
implementuje přenositelné uživatelské rozhraní pomocí knihovny libSDL a
obsahuje podporu pro lazení emulovaného programu, nebo hru více hráčů po síti
TCP/IP.

Při následující analýze a návrhu vlastního emulátoru je kladen důraz na
potlačení uvedených nedostatků při zachování předností existujících řešení.

Emulátor bude navržen modulárně, aby bylo v budoucnu snadné ho rozšířit
o podporu emulace dalších videoherních systémů při zachování společného
ovladání. Stejně tak podpora platforem nebude způsobem implementace omezena
pouze na operační systémy osobních počítačů.
