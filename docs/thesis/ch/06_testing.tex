%%%%%%%%%%%%%%%%%%%%%%%%%%%%%%%%%%%%%%%%%%%%%%%%%%%%%%%%%%%%%%%%%%%%%%%%%%%%%%%
% bc.tex - Bachelor thesis                                                    %
% Subject: bc_emu - portable video game emulator                              %
% Chapter: Uvod                                                               %
% Author: Ondrej Balaz <ondra@blami.net>                                      %
%%%%%%%%%%%%%%%%%%%%%%%%%%%%%%%%%%%%%%%%%%%%%%%%%%%%%%%%%%%%%%%%%%%%%%%%%%%%%%%

\chapter{Testování a lazení}\label{chap:test}

Emulace videoherního systému je složitý proces sestávající z napodobování řady
funkčních součástí tohoto systému, včetně replikace jejich nedokonalostí, nebo
dokonce, chyb. Dvojnásobně toto platí o emulátorech videoherních konzolí, kdy
ve většině případů není dostupná žádná oficiální specifikace hardwaru a
neoficiální se často liší od skutečnosti, je mylná nebo vůbec neexistuje.

Od samotného počátku implementace vede k prvním viditelným výsledkům často
dlouhá cesta, na níž se nelze obejít bez důkladného odlazení a otestování
jednotlivých částí kódu, které jsou stavebními kameny výsledného programu.
Vzhledem ke způsobu práce emulátoru mohou i malé chyby snadno způsobit
řetězovou reakci ústící v neočekávané chování na místě, kde chyba vůbec
nenastala. Není třeba zmiňovat, že takové chyby se hledají nejhůře.

Tato kapitola v sekci~\ref{chap:test_debug} stručně popisuje metody jakými byl
lazen a testován vyvíjený program a v sekci~\ref{chap:test_results} shrnuje
výsledky testování současné implementace.

% -----------------------------------------------------------------------------
% Prostredky a zpusob lazeni
% -----------------------------------------------------------------------------

\section{Prostředky a způsob lazení}\label{chap:test_debug}

Interpretační způsob implementace emulátoru výrazně usnadňuje lazení programu
tím, že jsou všechny části systému představovány datovými strukturami
reprezentujícími jejich vnitřní stav (obsah interní paměti, registrů apod.) a
kód emulovaného programu je zpracováván v iteracích hlavní programové smyčky.

V průběhu vývoje je tak možné \uv{sledovat} dění uvnitř emulovaného systému
pomocí běžných ladících nástrojů umožňujících krokování a sledování obsahu
proměnných, jakým je například debugger GDB~\cite{wwwGDB}.

Kromě toho je možné využít některých ladících mechanismů implementovaných přímo
v programu. Jedním z nich je například funkce zpětného volání implementovaná
při obsluze přerušení.

Veškeré důležité informace lze v každém kroku programu vypsat pomocí ladícího
makra {\it debug()}, jehož použití je podmíněno sestavením s aktivovanou
proměnnou {\tt DEBUG} sestavovacího systému CMake. Aktivace této proměnné navíc
způsobí, že program bude na standardní chybový výstup vypisovat řadu informací
z důležitých míst v kódu (např. upozornění na provedení neošetřeného přístupu
do paměti, pokus o vykonání ilegální instrukce atd.)

K pohodlnosti lazení také přispívá modulární architektura programu, která
dovoluje jednotlivé moduly, nebo jejich části oddělit a ladit zvlášť. Tento
způsob byl využit zejména při implementaci emulace CPU HuC6280, která je z
hlediska funkčnosti nejdůležitější.

I přes všechny tyto skutečnosti může být lazení programu někdy problematické a
hodilo by se mít integrovaný ladící nástroj, který by umožňoval alespoň
sledování registrů a obsahu paměti během krokování emulovaného programu.
Implementace takového nástroje dostatečně obecného vzhledem k architektuře a
vizi dalšího rozšiřování programu o nové moduly emulátorů je rozhodně nad rámec
této práce.

% -----------------------------------------------------------------------------
% Testovani funkcnosti
% -----------------------------------------------------------------------------

\section{Testování}\label{chap:test_results}

V rané fázi vývoje probíhalo testování (hlavně implementace CPU HuC6280) pomocí
miniaturních assemblerových programů přeložených pomocí software
MagicKit~\cite{wwwMagicKit} a porovnání výsledků proti běhu stejného programu v
mírně modifikovaném emulátoru M6502~\cite{wwwM6502}.

Další testy byly provedeny sadou ROM obrazů pořízených z paměti čipových karet
HuCard speciálním zařízením PCE Pro 32M. To umožňuje záznam obrazu ROM
vloženého modulu HuCard pomocí rozhraní USB do binárního souboru.

Toto testování proběhlo na řadě odlišných obrazů ROM ve velikostech od 257~KB
do 1~MB, kdy byl porovnán průběh programu v programu {\em bc\_emu} proti
průběhu téhož programu monitorovanému ladícím nástrojem emulátoru
Mednafen~\cite{wwwMednafen}.

V průběhu testování pomocí obrazů ROM původních herních programů byla objevena
a opravena řada chyb a nesrovnalostí v emulaci CPU HuC6280. Během provádění
oprav byly na řadu citlivých míst (zápisy do paměti, vstupně-výstupní stránky
apod.) přidány ladící hlášky, které umožňují sledovat podezřelé zápisy, nebo
čtení z neexistujících adres a určit tak příčinu nefunkčnosti programu.

Stále problematickou oblastí je emulace VDC a VCE. I přes to, že byla řada chyb
opravena, je spouštění některých herních programů poznamenáno chybami v
obrazovém výstupu (artefakty, chybějící sprajty). Řada těchto chyb je
způsobena, z hlediska specifikace, nekorektním chováním programu (např. herní
program {\em Pc Genjin 2} se snaží do barevných palet zapisovat hodnoty barev
delší než $3 * 3$ bajty, což bylo třeba ošetřit použitím pouze spodních 9-ti
bajtů apod.).

I přes uvedené skutečnosti je v případě velké části obrazů ROM, které bylo
možné otestovat, emulátor plně funkční a herní programy se dají bez větších
problémů používat, např.:
{\em Doraemon Meikyu Daisakusen}, {\em Makai Prince Dorabo Chan}, {\em Pc
Genjin 2}, {\em Son Son}, {\em Xevious} nebo {\em F1 Circus}.

Za funkční jsou považovány i programy které trpí některou z
následujících kosmetických vad, vyplývajích z částečné, nebo zcela zanedbané
implementace nekritických částí specifikace. Tyto chyby lze lehce opravit v
krátkém čase:

\begin{itemize}
\item Řada herních programů (např. {\em Mesopotamia}) využívá DDA režim PSG.
	Ten není v této implementaci programu {\em bc\_emu} podporován, takže
	emulace postrádá zvukový výstup.

\item Některé herní programy (např. {\em Shinobi}) používají efekt paralaxního
	skrolování obrazu implementovaný pomocí registru {\sf BXR} pomocného
	grafického procesoru VDC. Práce s tímto registrem je naimplementována
	chybně a proto je obrazový výstup programu nesprávný.

\item Některé herní programy využívají kolizi sprajtu 0, která v současné době
	není implementována (lze pozorovat např. v pokročilejší fázi programu {\em
	Doraemon Nobita no Dorabitan Night}).

\item Řada herních programů využívá k příznaku priority sprajtu (SPBG), která v
	současné době není implementována (např. {\em Soukoban World}).

\item Některé herní programy využívají instrukci {\sc set}, jejíž implementace
	je v některých případech nefunkční.
\end{itemize}

Vzhledem k tomu, že vývoj programu proběhl za dobu uplynulých dvou semestrů a
podporovaná množina funkcí videoherního systému NEC PCEngine je oproti
existujícím emulátorům velmi malá, nemá smysl porovnávat řešení z hlediska
výkonu, nebo právě podpory jednotlivých funkcí a rozšíření systému NEC PCEngine.

%
% Platformy
%

\subsection{Platformy}

Překlad a testování programu proběhlo s výše uvedenými výsledky na všech
platformách, kde byla požadována funkčnost programu:

\begin{itemize}
\item GNU/Linux, distribuce Debian a RedHat Enterprise Linux 5 (x86-64 i i386)
\item Microsoft Windows XP SP3 (i386)
\end{itemize}

Navíc bylo ověřeno, že program je po drobných úpravách možné přeložit ve
vývojovém prostředí devkitPro~\cite{wwwDevkitPro} pro videoherní handheld
Nintendo DS. Tato platforma není podporována knihovnou SDL, takže bez
implementace specifického modulu uživatelské rozhraní nebylo možné skutečně
ověřit funkčnost.

K překladu bylo ve všech případech využito překladačů ze sady
GCC~\cite{wwwGCC}, nebo klonů na nich založených. Vzhledem k tomu, že je
program psán tak, aby byla dodržena norma ANSI C, měl by být zaručen i překlad
(vyžadující maximálně kosmetické úpravy) jinými překladači implementujícími
tuto normu.
