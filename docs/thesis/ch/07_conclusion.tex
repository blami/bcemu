%%%%%%%%%%%%%%%%%%%%%%%%%%%%%%%%%%%%%%%%%%%%%%%%%%%%%%%%%%%%%%%%%%%%%%%%%%%%%%%
% bc.tex - Bachelor thesis                                                    %
% Subject: bc_emu - portable video game emulator                              %
% Chapter: Zaver                                                              %
% Author: Ondrej Balaz <ondra@blami.net>                                      %
%%%%%%%%%%%%%%%%%%%%%%%%%%%%%%%%%%%%%%%%%%%%%%%%%%%%%%%%%%%%%%%%%%%%%%%%%%%%%%%

\chapter{Závěr}

Na základě dostupných informací byla nastudována architektura videoherního
systému NEC PCEngine a získané poznatky byly shrnuty do ucelené technické
specifikace uvedené v kapitole~\ref{chap:spec}. Ta posloužila jako základ pro
analýzu, návrh a následnou implementaci emulátoru tohoto systému.

S ohledem na modularitu ve smyslu možnosti rozšíření o podporu dalších
videoherních systémů, nebo uživatelských rozhraní, byl naimplementován emulátor
systému NEC PCEngine, jehož funkčnost byla úspěšně ověřena pomocí řady
původních herních programů.

Emulátor zahrnuje alespoň základní podporu všech částí základní varianty
systému NEC PCEngine a interpretačním způsobem umožňuje spouštění a provádění
herních programů uložených v souborech představujících obsah paměti ROM
původních čipových karet HuCard.

Zvolená architektura programu odděluje veškerou logiku emulace a interakce s
uživatelem do diskrétních modulů s pevně definovaným rozhraním, čímž umožňuje
nejen snadné rozšíření o podporu emulace dalších videoherních systému, ale i
zjednodušení přenositelnosti zapouzdřením kódu závislého na platformě.

Vzhledem k tomu, že program splňuje všechny zadáním specifikované požadavky a
nesrovnalosti, které se projevují při provádění emulovaných programů, jsou buď
kosmetické a lze je bez výraznějších zásahů do architektury programu odstranit,
nebo jsou způsobeny použitím nestandartních technik v rámci těchto programů a 
je třeba je řešit individuálně, lze prohlásit, že cíl této práce byl splněn.

% -----------------------------------------------------------------------------
% Dalsiho vyvoj
% -----------------------------------------------------------------------------

\section{Další vývoj}

Díky modulární architektuře umožňující implementaci podpory dalších
videoherních systémů a uživatelských rozhraní jsou možnosti rozšiřování téměř
nekonečné. Před samotným rozšiřováním tímto směrem by však mělo být zváženo
několik zásadnějších zásahů do architektury programu.

Do programu by mělo být přidáno obecné rozhraní umožňující modulům dotazovat se
na hodnoty konfiguračních klíčů (např. moduly uživatelského rozhraní by měly
mít možnost konfigurce vstupních zařízení, rozlišení obrazovky, formátu zvuku
apod.).

Dalším možným rozšířením je implementace nového typu modulu - ladícího
nástroje pro emulovaný program, který by uměl na obecné úrovni (samozřejmě by
autor modulu emulátoru musel provést nastavení) sledovat registry a paměťové
oblasti emulovaného systému, případně krokovat program.

Implementace ladícího nástroje by jistě pomohla při dalším zdokonalování modulu
emulátoru systému NEC PCEngine. Kromě odstranění drobných chyb a doimplementace
chybějících, v kapitole~\ref{chap:test_results} zmíněných, nekritických částí
specifikace, by bylo vhodné zdokonalit kód PSG a přidat podporu pro některá
populární rozšíření.

Po zásadnějších úpravách a stabilizaci architektury může být zajímavou výzvou
přenést program {\em bc\_emu} na videoherní handheld Nintendo DS. S drobnými
úpravami je už nyní možné pro tento systém přeložit jádro programu. Dalším
krokem je nastudování programového rozhraní knihoven devkitPro a implementace
modulu uživatelského rozhraní. Vzhledem k výkonu procesoru tohoto systému mohou
nastat potíže s výkonem. Dalším směrem vývoje proto může být optimalizace
emulačního kódu při zachování přenositelnosti.
