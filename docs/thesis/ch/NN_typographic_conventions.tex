%%%%%%%%%%%%%%%%%%%%%%%%%%%%%%%%%%%%%%%%%%%%%%%%%%%%%%%%%%%%%%%%%%%%%%%%%%%%%%%
% bc.tex - Bachelor thesis                                                    %
% Subject: bc_emu - portable video game emulator                              %
% Chapter: Typograficke konvence                                              %
% Author: Ondrej Balaz <ondra@blami.net>                                      %
%%%%%%%%%%%%%%%%%%%%%%%%%%%%%%%%%%%%%%%%%%%%%%%%%%%%%%%%%%%%%%%%%%%%%%%%%%%%%%%

\cleardoublepage
\noindent
\chapter*{Typografické konvence}

V následujícím textu se vyskytuje řada významných klíčových slov. Pro vyšší
přehlednost jsou použity následující typografické konvence:

\begin{itemize}
\item identifikátory a názvy funkcí budou sázeny {\it kurzívou}

\item adresy, offsety a datové konstanty budou uvedeny v hexadecimální soustavě
	sázeny {\tt strojopisem} s použitím prefixu {\tt \$} (v souladu s
	assemblerem pro procesory WDC 65c02 a HuC6280)

\item názvy registrů budou sázeny {\sf groteskem}

\item jména instrukcí budou sázena {\sc kapitálkami}

\item názvy souborů budou sázeny {\tt strojopisem}, v případě uvedení cesty,
	v UNIXové notaci vzhledem ke kořenovému adresáři projektu
\end{itemize}
